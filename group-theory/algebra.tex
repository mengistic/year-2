
\PassOptionsToPackage{dvipsnames,table,usenames,svgnames}{xcolor}
\documentclass[a4paper, 10pt, openright,twoside,onecolumn]{memoir}

\usepackage{asymptote}
\def\asydir{asydir}

\usepackage{marginnote}
\def\blank#1{\rule{#1}{0.5pt}}


\usepackage{lipsum}
\usepackage{tikz}
\usepackage{xcolor}

% ==> language
\usepackage[no-math]{fontspec}
\usepackage{fontawesome}
\usepackage[Khmer, Latin]{ucharclasses}
\usepackage{etoolbox}
\usepackage{enumitem}

% ==> main font
\setmainfont[
	BoldFont = Kdam Thmor,
	ItalicFont = Khmer OS Metal Chrieng,
	BoldItalicFont = Khmer OS Muol,
	SmallCapsFont = TeX Gyre Pagella,
	Script=Khmer,Scale=1
]{Khmer OS Siemreap}

\XeTeXlinebreaklocale "kh"
\XeTeXlinebreakskip = 0pt plus 1pt minus 1pt


\newfontfamily{\khmerfamily}
[
	BoldFont = Kdam Thmor,
	ItalicFont = Khmer OS Metal Chrieng,
	BoldItalicFont = Khmer OS Muol,
	SmallCapsFont = TeX Gyre Pagella,
	Script=Khmer,Scale=1
]{Khmer OS Siemreap}
\newfontfamily{\englishfamily}{TeX Gyre Pagella}[Ligatures=TeX, Scale=1.1]
\newfontfamily{\monofamily}[Scale=1.3]{Latin Modern Mono}
\newfontfamily{\latinfamily}[Scale=1.3]{Latin Modern Roman}
\newfontfamily{\tacteingfamily}[Scale=3]{Tacteing}

% Define new font family (have to)   <-- BUG
\newrobustcmd{\englishfont}{\englishfamily\let\currentenglish\englishfamily }
\newrobustcmd{\khmerfont}{\khmerfamily\let\currentkhmer\khmerfamily}
\newrobustcmd{\mono}{\monofamily\let\currentenglish\monofamily }
\newrobustcmd{\en}{\latinfamily\let\currentenglish\latinfamily}
\newrobustcmd{\tacteing}{\tacteingfamily\let\currentenglish\tacteingfamily}

% Initialize fonts
\khmerfont\englishfont
\setTransitionsForLatin{\currentenglish}{\currentkhmer}

% Change typewriter font family
\renewcommand{\ttfamily}{\mono}

% Fixed: math font substitution, if you use |mathpazo|
\let\temp\rmdefault
\usepackage{mathpazo}
\let\rmdefault\temp
% <==
% ==> khmer fonts
\newcommand{\kml}{
	\fontspec[ Script=Khmer, Scale=1,AutoFakeBold=1, AutoFakeSlant=0.25] 
	{Khmer OS Muol}\selectfont
}
\newcommand{\kpali}{
	\fontspec[ Scale=1, Script=Khmer,AutoFakeBold=1, AutoFakeSlant=0.25] 
	{Khmer OS Muol Pali}\selectfont
}
\newcommand{\kbk}{
	\fontspec[Script=Khmer, Scale=1.125,AutoFakeBold=1, AutoFakeSlant=0.25]
	{Khmer OS Bokor}\selectfont
}
\newcommand{\kmetal}{
	\fontspec[Script=Khmer, Scale=1.125,AutoFakeBold=1, AutoFakeSlant=0.25]
	{Khmer OS Metal Chrieng}\selectfont
}


% useful khmer-math shortcuts
\def\KHstop{\quad\text{។}}
\def\KHor{\quad\text{ឬ}\quad}
\def\KHand{\quad\text{និង}\quad}
% <==
% ==> khmer counter
%khmer number

\makeatletter
\def\khmer#1{\expandafter\@khmer\csname c@#1\endcsname}
\def\@khmer#1{\expandafter\@@khmer\number#1\@nil}
\def\@@khmer#1{%
	\ifx#1\@nil% terminate when encounter @nil
	\else%
	\ifcase#1 ០\or ១\or ២\or ៣\or ៤\or ៥\or ៦\or ៧\or ៨\or ៩\fi%
	\expandafter\@@khmer% recursively map the following characters
	\fi}

% khmer alphabet
\def\khmernumeral#1{\@@khmer#1\@nil}
\def\alpkh#1{\expandafter\@alpkh\csname c@#1\endcsname}
\def\@alpkh#1{%
	\ifcase#1% zero -> none
	\or ក\or ខ\or គ\or ឃ\or ង%
	\or ច\or ឆ\or ជ\or ឈ\or ញ%
	\or ដ\or ឋ\or ឌ\or ឍ\or ណ%
	\or ត\or ថ\or ទ\or ធ\or ន%
	\or ប\or ផ\or ព\or ភ\or ម%
	\or យ\or រ\or ល\or វ\or ស%
	\or ហ\or ឡ\or អ%
	\else%[most]
	\@ctrerr % otherwise, counter error!
	\fi}
\makeatother

%declare new enumerate counter
\AddEnumerateCounter{\alpkh}{\@alpkh}{ឈ}
\AddEnumerateCounter{\khmer}{\@khmer}{៣}
% <==

% <==
% ==> redefine the names
\def\chaptername{ជំពូកទី}
\def\axiomName{ស្វ័យសត្យ}
\def\definitionName{និយមន័យ}
\def\theoremName{ទ្រឹស្តីបទ}
\def\lemmaName{បទគន្លឹះ}
\def\propositionName{សំណើ}
\def\corollaryName{វិបាក}
\def\exampleName{ឧទាហរណ៍}
\def\exerciseName{លំហាត់}
\def\proofName{សម្រាយបញ្ជាក់}
\def\solutionName{ដំណោះស្រាយ}
\def\remarkName{សម្គាល់}
% <==
% ==> math
\usepackage{amsmath, amsthm, amsfonts, amssymb}
\usepackage{mathtools}

\mathtoolsset{centercolon} % not work when using |mathpazo|
\DeclarePairedDelimiter\abs{\lvert}{\rvert}
\DeclarePairedDelimiterX\norm[1]\lVert\rVert{
	\ifblank{#1}{\:\cdot\:}{#1}
}
\def\set#1#2{\left\{#1 ~:~ #2\right\}}
\def\permil{\text{\hskip 0.3pt\englishfont\textperthousand}}


\DeclareMathOperator{\arccot}{cot}
\DeclareMathOperator{\arcsec}{arcsec}
\DeclareMathOperator{\arccsc}{arccsc}
\DeclareMathOperator{\lcm}{lcm}
\DeclareMathOperator{\ord}{ord}
\DeclareMathOperator{\sym}{sym}
\DeclareMathOperator{\tr}{trace}
\DeclareMathOperator{\dom}{dom}
\DeclareMathOperator{\ran}{ran}



\def\N{\mathbb{N}}
\def\Z{\mathbb{Z}}
\def\Q{\mathbb{Q}}
\def\Qc{\mathbb{Q}^{\complement}}
\def\R{\mathbb{R}}
\def\C{\mathbb{C}}
\def\F{\mathbb{F}}
\def\K{\mathbb{K}}
\def\P{\mathbb{P}}
\def\labelitemi{$\circ$}
\def\inv{^{-1}}
\def\ang#1{\left\langle#1\right\rangle}
\def\tran{^\mathrm{T}}
\renewcommand{\vec}[1]{\mathbf{#1}}
% <==
% ==> listings
\usepackage{xcolor}
\usepackage{listings}
% ==> basic
\lstset{%
	basicstyle=\small\ttfamily,
	keywordstyle=\color{black},
	commentstyle=\color{gray},
	keywordstyle=[1]{\color{blue!90!black}},
	keywordstyle=[2]{\color{magenta!90!black}},
	keywordstyle=[3]{\color{red!60!orange}},
	keywordstyle=[4]{\color{teal}},
  keywordstyle=[5]{\color{magenta!90!black}}, % <-- otherkeywords, BUG
	commentstyle=\color{gray},
	stringstyle=\color{green!60!black},
	tabsize=2,
	%
	numbers=left,
	numberstyle=\tiny\color{blue!70!gray},
	stepnumber=1,
	%
	frame=Lt,
	breaklines=true,
	xleftmargin=0cm,
	rulecolor=\color{gray!50!black},
	aboveskip=0.5cm,
	belowskip=0.5cm
}
% <==
% ==> code c
\lstdefinelanguage{cmeng}{
  morekeywords={
    auto,break,case,char,const,continue,default,do,double,%
    else,enum,extern,float,for,goto,if,int,long,register,return,%
    short,signed,sizeof,static,struct,switch,typedef,union,unsigned,%
    void,volatile,while},%
  morekeywords=[2]{
    printf, scanf,  include
  },
  otherkeywords={\#,<,>,\&},
  morekeywords=[5]{\#,<,>,\&},
  sensitive,%
	morecomment=[l]{//},
	morecomment=[s]{/*}{*/},
	morestring=[b]',
	morestring=[b]",
}
% <==
% ==> code python
\lstdefinelanguage{py}{
	morekeywords={
		access,and,as,break,class,continue,def,del,elif,else,
		except,exec,finally,for,from,global,if,import,in,is,lambda,
		not,or,pass,print,raise,return,try,while},
	% Built-ins
	morekeywords=[2]{
		abs,all,any,basestring,bin,bool,bytearray,
		callable,chr,classmethod,cmp,compile,complex,delattr,dict,dir,
		divmod,enumerate,eval,execfile,file,filter,float,format,
		frozenset,getattr,globals,hasattr,hash,help,hex,id,input,int,
		isinstance,issubclass,iter,len,list,locals,long,map,max,
		memoryview,min,next,object,oct,open,ord,pow,property,range,
		raw_input,reduce,reload,repr,reversed,round,set,setattr,slice,
		sorted,staticmethod,str,sum,super,tuple,type,unichr,unicode,
		vars,xrange,zip,apply,buffer,coerce,intern,True,False},
	%
	morecomment=[l]\#,%
	morestring=[b]',%
	morestring=[b]",%
	morecomment=[s]{'''}{'''},% used for documentation text
	%                         % (mulitiline strings)
	morecomment=[s]{"""}{"""},% added by Philipp Matthias Hahn
	morestring=[s]{r'}{'},% `raw' strings
	morestring=[s]{r"}{"},%
	morestring=[s]{r'''}{'''},%
	morestring=[s]{r"""}{"""},%
	morestring=[s]{u'}{'},% unicode strings
	morestring=[s]{u"}{"},%
	morestring=[s]{u'''}{'''},%
	morestring=[s]{u"""}{"""},%
	%
	sensitive=true,%
}
% <==
% ==> code asy
\lstdefinelanguage{asy}{ %% Added by Sivmeng HUN
	morekeywords=[1]{
		import, for, if, else,new, do,and, access,
		from, while, break, continue, unravel, 
		operator, include, return},
	morekeywords=[2]{
		struct,typedef,static,public,readable,private,explicit,
		void,bool,int,real,string,var,picture,
		pair, path, pair3, path3, triple, transform, guide, pen, frame
	},
	morekeywords=[3]{
		true,false,and,cycle,controls,tension,atleast,
		curl,null,nullframe,nullpath,
		currentpicture,currentpen,currentprojection,
		inch,inches,cm,mm,pt,bp,up,down,right,left,
		E,N,S,W,NE,NW,SE,SW,
		solid,dashed,dashdotted,longdashed,longdashdotted,
		squarecap,roundcap,extendcap,miterjoin,roundjoin,
		beveljoin,zerowinding,evenodd,invisible
	},
	morekeywords=[4]{
		size,unitsize,draw,dot,label,
		sqrt,sin,cos,tan,cot,Sin,Cos,Tan,Cot,
		graph,
	},
	%
	morecomment=[l]{//},
	morecomment=[s]{/*}{*/},
	morestring=[b]',
	morestring=[b]",
	%
}
% <==
% <==
% ==> mdframe theorem
\usepackage{thmtools}
\usepackage[framemethod=TikZ]{mdframed}

% ==> Axiom
\mdfdefinestyle{mdAxiomStyle}{
	roundcorner = 10pt,
	linewidth=1.125pt,
	skipabove=12pt,
	innerbottommargin=9pt,
	skipbelow=2pt,
	nobreak=true,
	linecolor=blue,
	backgroundcolor=TealBlue!5,
}
\declaretheoremstyle[
	headfont=\bfseries\color{MidnightBlue},
	bodyfont=\itshape,
	mdframed={style=mdAxiomStyle},
	headpunct={\\[3pt]},
	postheadspace={0pt},
	notefont=\bfseries\itshape\color{gray!50!blue},
	notebraces={~(}{)},
]{mdAxiom}
% <==
% ==> Definition
\mdfdefinestyle{mdDefinitionStyle}{
	roundcorner = 10pt,
	linewidth=1.125pt,
	skipabove=12pt,
	innerbottommargin=9pt,
	skipbelow=2pt,
	nobreak=true,
	linecolor=blue,
	backgroundcolor=TealBlue!5,
}
\declaretheoremstyle[
	headfont=\bfseries\color{MidnightBlue},
	bodyfont=\itshape,
	mdframed={style=mdDefinitionStyle},
	headpunct={\\[3pt]},
	postheadspace={0pt},
	notefont=\bfseries\itshape\color{gray!50!blue},
	notebraces={~(}{)},
]{mdDefinition}
% <==
% ==> Theorem
\mdfdefinestyle{mdTheoremStyle}{%
	linewidth=1pt,
	skipabove=12pt,
	frametitleaboveskip=5pt,
	frametitlebelowskip=0pt,
	skipbelow=2pt,
	innertopmargin=5pt,
	innerbottommargin=8pt,
	nobreak=true,
	linecolor=RawSienna!40,
	backgroundcolor=Salmon!10,
}
\declaretheoremstyle[
	headfont=\bfseries\color{RawSienna},
  headindent=0pt,
	bodyfont=\itshape,
	mdframed={style=mdTheoremStyle},
	notefont=\normalfont\itshape\small\color{RawSienna},
	notebraces={~$\big\langle$}{$\big\rangle$},
	headpunct={\\[3pt]},
	postheadspace={0pt},
]{mdTheorem}
% <==
% ==> Exercise
\mdfdefinestyle{mdExerciseStyle}{%
	skipabove=8pt,
	skipbelow=2pt,
	linewidth=1pt,
	rightline=true,
	leftline=true,
	topline=true,
	bottomline=true,
	linecolor=ForestGreen!20,
	backgroundcolor=ForestGreen!20
}
\declaretheoremstyle[
	headfont=\bfseries\color{ForestGreen!70!black},
	bodyfont=\normalfont,
	headindent=0pt,
	spaceabove=2pt,
	spacebelow=2pt,
	mdframed={style=mdExerciseStyle},
	notefont=\small\bfseries\itshape\color{DarkGreen},
	notebraces={~(}{)},
	headpunct={ },
]{mdExercise}
% <==
% ==> Proof
\declaretheoremstyle[
	spaceabove=6pt,
	spacebelow=6pt,
	headindent=0pt,
	headfont=\normalfont\color{magenta!70!blue},
	bodyfont = \normalfont,
	postheadspace=1em,
	qed=$\color{magenta!70!black}\blacksquare$,
	headpunct={ },
]{mdProof}
% <==

% ==> Declaring the theorems
\declaretheorem{axiom}[style=mdAxiom, name={\kml\axiomName}, numberwithin=chapter]
\declaretheorem[style=mdDefinition, name={\kml\definitionName}, sibling=axiom]{definition}

\declaretheorem{theorem}[style=mdTheorem, name={\theoremName},numberwithin=chapter]
\declaretheorem{lemma}[style=mdTheorem, name={\lemmaName},sibling=theorem]
\declaretheorem{proposition}[style=mdTheorem, name={\propositionName}, sibling=theorem]
\declaretheorem{corollary}[style=mdTheorem, sibling=theorem, name={\corollaryName}]

\declaretheorem{example}[name={\exampleName},style=mdExercise, numberwithin=chapter]
\declaretheorem{exercise}[name={\exerciseName},style=mdExercise, sibling=example]

%% Proof
\declaretheorem[name={\bfseries\proofName},numbered=no,style=mdProof]{Proof}
\renewenvironment{proof}{\begin{Proof}}{\end{Proof}}
\declaretheorem[name={\itshape\solutionName},numbered=no,style=mdProof]{solution}
% <==
% <==
% ==> tweaking memmoir
\checkandfixthelayout
%\setlrmarginsandblock{2cm}{8cm}{*}
%\setulmarginsandblock{3cm}{*}{1.5}


\makepagestyle{myruled}
\makeheadrule {myruled}{\textwidth}{\normalrulethickness}
\makefootrule {myruled}{\textwidth}{\normalrulethickness}{\footruleskip}
\makeevenhead {myruled}{}{\small\itshape\leftmark} {}
\makeoddhead  {myruled}{}{\small\itshape\rightmark}{}
\makeevenfoot {myruled}{}{\small\thepage} {}
\makeoddfoot  {myruled}{}{\small\thepage} {}

\makeatletter % because of \@chapapp
\makepsmarks{myruled}{
  \nouppercaseheads
  \createmark{chapter}{both}{shownumber}{}{. \space}
  \createmark{section}{right}{shownumber}{}{. \space}
  \createmark{subsection}{right}{shownumber}{}{. \space}
  \createmark{subsubsection}{right}{shownumber}{}{. \ }
  \createplainmark {toc} {both} {\contentsname} 
  \createplainmark {lof} {both} {\listfigurename}
  \createplainmark {lot} {both} {\listtablename} 
  \createplainmark {bib} {both} {\bibname} 
  \createplainmark {index} {both} {\indexname} 
  \createplainmark {glossary} {both} {\glossaryname}
}
\makeatother
\setsecnumdepth{subsubsection}
\pagestyle{myruled}

\renewcommand{\headheight}{17pt}
% <==

\title{ពិជគណិត}
\author{បង្រៀនដោយលោកគ្រូ ចាន់ វិធូ}
\date{២៣ ធ្នូ ២០២១}

\begin{document}

\chapter{ក្រុម}

\section{និយមន័យក្រុម}

\subsection{ធាតុច្រាស}
សំណុំ $E$ ប្រដាប់ដោយម្មាណវិធីក្នុង $\cdot$ ហើយ $e\in E$ ជាធាតុណឺត និង
$a\in E$ ។ គេថា $a\inv\in E$ ជាធាតុច្រាសនៃ $a$ កាលណា $aa\inv = a\inv a=e$ ។
\begin{example}
លើ $\Z,~\Q,~\R$ ចំពោះប្រមាណវិធី $+$ ធាតុច្រាសនៃ $a$ គឺ $-a$ ។
\end{example}
\begin{example}
លើ $\Z,~\Q,~\R$ ចំពោះប្រមាណវិធី $\times$ នោះធាតុច្រាសនៃ $a$ គឺ $\frac{1}{a}$
កាលណា $a\neq 0$ ។
\end{example}
\begin{theorem}
លើ $E$ ប្រដាប់ដោយប្រមាណវិធីក្នុង $\cdot$ ហើយ $a,b$ មានធាតុច្រាសគេបាន
\[(ab)\inv=a\inv b\inv\]
\end{theorem}
\begin{proof}
យើងមាន  $a,b\in E$ នោះ $ab\in E$ ។ ពិនិត្យ
\[ab\cdot b\inv a\inv=a (bb\inv) a\inv=aea\inv=aa\inv=e\] 
និង
\[b\inv a\inv\cdot ab=b\inv (a\inv a)b=b\inv eb=b\inv b=e\]
នោះយើងបាន $b\inv a\inv$ ជាធាតុច្រាសនៃ $ab$ ។ 
ដូចនេះ $\boxed{(ab)\inv = b\inv a\inv}$ ។
\end{proof}

\subsection{ផ្នែកស្តាប (closed)}
សំណុំ $E$ ប្រដាប់ដោយប្រមាណវិធីក្នុង $\cdot$ ។ $A$ ជាផ្នែកនៃ $E$ ($A\subset E$) ។
គេថា $A$ ជាផ្នែកស្តាបនៃ $E$ លុះត្រាតែ $ab\in A$ ចំពោះគ្រប់ $a,b\in A$ ។

\section{ប្រមាណវិធីក្រៅ}
គេឱ្យ $E,~E$ ជាសំណុំ ។ យកអនុគមន៍
\[
\begin{array}{lrll}
\ast ~: & K\times E &\to & E\\
     & (\lambda, x) &\mapsto & \lambda\ast x
\end{array}
\]
យើងហៅ $\ast$ ជាប្រមាណវិធីក្រៅលើ $E$ ការីក្នុង $K$ ។

\section{ក្រុម}
\subsection{កន្លះក្រុម (semigroup)}
\begin{definition}
គេឱ្យ $G$ ជាសំណុំប្រមាណវិធីក្នុង $\cdot$ ។ គេថា​ $(G,~\cdot)$ 
ជាកន្លះក្រុមកាលណាប្រមាណវិធីនេះមានលក្ខណៈផ្តុំ (associativity) ។ មានន័យថាគ្រប់
$a,b,c\in G$ យើងបាន $(ab)c=a(bc)$ ។
\end{definition}

\begin{example}
យើងមាន $(\N,+),~(\Z,+),~(\Q,+)$ និង $(\R,+)$ ជាកន្លះក្រុម ។
\end{example}
\begin{example}
យើងមាន $(\N,\cdot),~(\Z,\cdot),~(\Q,)$ និង $(\R,\cdot)$ ជាកន្លះក្រុម ។
\end{example}
\begin{example}
$(\mathcal{P}(E),~\cup),~(\mathcal{P}(E),~\cap),~$ ជាកន្លះក្រុម
\end{example}

\newpage
\begin{example}
លើ $\Z$ ប្រដាប់ដោយប្រមាណវិធី $\ast$ កំណត់ដោយ
\[a\ast b=a+b-ab\]
គ្រប់ $a,b\in\Z$ ។ បង្ហាញថា $(\Z,\ast)$ ជាកន្លះក្រុម ។
\end{example}
\begin{proof}
គ្រប់ $a,b,c\in\Z$ យើងមាន
\begin{align*}
(a\ast b)\ast c&=(a+b-ab)\ast c\\
&=(a+b-ab)+c-(a+b-ab)c\\
&=a+b+c-ab-bc-ca+abc
\end{align*}
និង
\begin{align*}
a\ast (b\ast c)&=a\ast (b+c-bc)\\
&=a+(b+c-bc) -a(b+c-bc)\\
&=a+b+c-ab-bc-ca+abc
\end{align*}
នោះយើងបាន $(a\ast b)\ast c=a\ast (b\ast c)$ ដូចនេះ
\fbox{$(\Z,\ast)$ ជាកន្លះក្រុម} ។
\end{proof}


\begin{example}
លើ $\R^{\ast}_{+}$ ប្រដាប់ដោយប្រមាណវិធីក្នុង $\ast$ កំណត់ដោយ
\[
a\ast b=a^{\ln b}
\]
គ្រប់ $a,b\in \R^{\ast}_{+}$ ។ បង្ហាញថា $(\R^{\ast}_{+}, \ast)$
ជាកន្លះក្រុម ។
\end{example}
\begin{proof}
គ្រប់ $a,b,c\in \R^{\ast}_{+}$ យើងមាន
\begin{align*}
(a\ast b)\ast c&=(a^{\ln b})\ast c\\
&= \left(a^{\ln b}\right)^{\ln c}\\
&= a^{(\ln b)(\ln c)}
\end{align*}
និង 
\begin{align*}
a*(b*c)&=a*(b^{\ln c})\\
&=a^{\ln (b^{\ln c})}\\
&=a^{(\ln b)(\ln c)}
\end{align*}
នោះយើងបាន $(a\ast b)\ast c=a\ast (b\ast c)$ ដូចនេះ
\fbox{$(\R^{\ast}_{+}, \ast)$ ជាកន្លះក្រុម} ។
\end{proof}

\newpage
គេឱ្យ $(G,*)$ ជាកន្លះក្រុម ហើយចំពោះ ​$x_1,x_2,\dots,x_n\in G$ គេតាង
\begin{align*}
\mathop{\ast}\limits_{i=1}^{n}&=x_1*x_1*\cdots*x_n\\
&=\left(\mathop{\ast}\limits_{i=1}^{k}\right)*
\left(\mathop{\ast}\limits_{i=k+1}^{n}\right)
\end{align*}
ហើយគេកំណត់សរសេរ
\[
x^n := \underbrace{x*x*\cdot *x}_{n \text{ដង}}
\]
នោះយើងបាន $(x^{n})^m=x^{}nm$ និង $x^n*x^m=x^{n+m}$ ។

\subsection{ម៉ូណូអុីត (Monoide)}
\begin{definition}
សំណុំ $G$ ប្រដាប់ដោយប្រមាណវិធីក្នុង $*$ ។ គេថា $(G,~*)$ ជាម៉ូណូអុីតកាល ណា 
$(G,~*)$ ជាកន្លះក្រុមហើយមានធាតុណឺត ។ នោះយើងបាន
\begin{itemize}
\item $a*(b*c)=(a*b)*c$ គ្រប់ $a,b,c\in G$ និង
\item មាន ​$e\in G$ ដែល $e*a=a*e=a$ គ្រប់ $a\in G$ ។
\end{itemize}
\end{definition}

\begin{example}
$(\N,+),~(\Z,+),~(\Q,+),~(\R,+)$ ជាម៉ូណូអុីត ។
\end{example}
\begin{example}
$(\N, \cdot), ~(\Z,\cdot),~(\Q,\cdot), ~(\R, \cdot)$ ជាម៉ូណូអុីត ។
\end{example}
\begin{example}
លើ $\R^{\ast}_{+}$ ប្រដាប់ដោយប្រមាណវិធីក្នុង $*$ កំណត់ដោយ
\[
a*b=a^{\ln b}
\]
ចំពោះគ្រប់ $a,b\in\R^{\ast}_{+}$ ។ តើ $(\R^{\ast}_{+},~*)$ ជាម៉ូណូអុីតឬទេ ។
\end{example}
\begin{proof}
តាមសម្រាយខាងលើ $(\R^{\ast}_{+},~*)$ ជាកន្លះក្រុម ។ ដើម្បីស្រាយថា 
$(\R^{\ast}_{+},~*)$ ជាម៉ូណូអុីតយើងត្រូវរក $x\in\R^{\ast}_{+}$ ដែល
$x*a=a*x=a$ គ្រប់ $a\in\R^{\ast}_{+}$ ។ យើងមាន
\[
a*e = a^{\ln e}=a
\] 
និង
\end{proof}
\begin{example}
$(\mathcal{P}(E),~\cup)$ និង $(\mathcal{P}(E),~\cup)$ ជាម៉ូណូអុីត ។
\end{example}

\subsection{ម៉ូណូអុីតរង}
\begin{definition}
គេឱ្យ $(G, ~*)$ ជាម៉ូណូអុីត និង $H\supset G$ ជាផ្នែកមិនទទេនៃ $G$ ។
គេថា $(H, ~*)$ ជាម៉ូណូអុីតរងប្រដាប់ដោយប្រមាណវិធី $*$ លុះត្រាតែបើធាតុ $e$ 
ជាធាតុណឺតនៃ $G$ នោះ $e\in H$ និង
\[a*b\in H\]
ចំពោះ $a,b\in H$ ។
\end{definition}
\begin{example}
គេយក $n\in\N$ នោះយើងបាន $(n\Z,~+)$ ជាម៉ូណូអុីតរងនៃ $(\Z, ~+)$​ ព្រោះ
$0$ ជាធាតុណឹតរបស់ $\Z$ ហើយ $0=0\cdot n\in n\Z$  នោះ $n\Z\neq\emptyset$ ។
ម្យ៉ាងទៀតបើ $a,b\in n\Z$ នោះមាន $x,y\in\Z$ ដែល $a=nx,~b=ny$ ហេតុនេះ
\[
a+b=nx+nx=n(x+y)\in n\Z
\]
នោះ $(n\Z,~+)$ ជាធាតុ
\end{example}
\begin{theorem}
ប្រសព្វនៃពីរម៉ូណូអុីតរងនៃម៉ូណូអុីត $(G,~*)$ ក៏ជាម៉ូណូអុីតរងនៃ $(G,~*)$ ដែរ ។
\end{theorem}
\begin{proof}
តាង $e$ ជាធាតុណឺតនៃ $G$ ហើយតាង $A,B\subseteq G$ ជាមូ៉ណូអុីតរងនៃ $(G,*)$ ។ 
នោះ $e\in A$ និង $e\in B$ ហេតុនេះ $e\in A\cap B$ ។

បន្ទាប់មកទៀតយើងស្រាយលក្ខណៈស្តាប ។ យក $x,y\in A\cap B$ នោះ
\begin{align*}
&\begin{cases}
x\in A\\
y\in A
\end{cases}
\KHand
\begin{cases}
x\in B\\
y\in B
\end{cases}\\
\implies\quad &x*y\in A\KHand x*y\in B
\end{align*}
នោះយើងបាន $x*y\in A\cap B$ ។
សរុបមកយើងបាន $(A\cap B, ~*)$ ជាម៉ូណូអុីតរងនៃ $(G,~*)$ ។
\end{proof}

\newpage
\begin{corollary}
ប្រសព្វនៃគ្រួសារម៉ូណូអុីតរងនៃ $(G,~*)$ ជាម៉ូណូអុីតរងនៃ $(G,*)$ ។
\end{corollary}
\begin{proof}
តាង $\mathcal{I}=\Big\{
	i\in\N ~:~ 
	A_i\subset G\text{ និង }
	(A_i,~*) \text{ ជាម៉ូណូអុីត}\Big\}$
ជាសំណុំសន្ទស្ស៍នៃគ្រូសារម៉ូណូអុីតរងនៃ $(G,~*)$ ។ យើងចង់ស្រាយថា
$A:=\bigcap_{i\in\mathcal{I}}$ ជាម៉ូណូអុីតរងដែរ ។

តាង $e$ ជាធាតុណឺតរបស់ $G$ ។ យើងឃើញថា $A\neq\emptyset$ ព្រោះ $e\in A_i$
គ្រប់ $i\in\mathcal{I}$ នោះយើងបាន $e\in\bigcap_{i\in\mathcal{I}}A_i=A$ ។
បន្ទាប់មកទៀតយើងស្រាយលក្ខណៈស្តាបលើ $A$ ។ យក $x,y\in A$ 
នោះគ្រប់ $i\in\mathcal{I}$ យើងបាន
\begin{align*}
&x\in A_i \wedge y\in A_i\\
\implies\quad &x*y\in A_i && (\text{លក្ខណៈស្តាបរបស់} A_i)\\
\implies\quad &x*y\in \bigcap_{i\in\mathcal{I}}A_i
\end{align*}
នោះ $(A,~*)$ មានធាតុណឹត $e$ ហើយមានលក្ខណៈស្តាបលើប្រមាណវិធី $*$ ។\\[0.3cm]
ដូចនេះ \quad\fbox{$(A,~+)$ ជាម៉ូណូអុីតរងនៃ $(G,~*)$ } ។

\end{proof}
%
\newpage
\noindent
\textbf{សម្គាល់៖} ប្រជុំនៃពីរម៉ូណូអុីតមិនមែនជាម៉ូណូអុីតទូទៅទេ ។ ឧទាហរណ៍
$(2\Z, +),~(3\Z,+)$ ជាម៉ូណូអុីតរងនៃ $(\Z,+)$ តែ $(2\Z\cup 3\Z, +)$
មិនមែនជាម៉ូណូអុីតទេព្រោះ $2\in 2\Z$ ហើយ $3\in 3\Z$ 
ក៏ប៉ុន្តែ $5=2+3\notin	 2\Z\cup 3\Z$ ។









































\end{document}