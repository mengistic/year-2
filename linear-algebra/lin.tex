\PassOptionsToPackage{dvipsnames,table,usenames,svgnames}{xcolor}
\documentclass[a4paper, 10pt, openright,twoside,onecolumn]{memoir}

\usepackage{asymptote}
\def\asydir{asydir}

\usepackage{marginnote}
\def\blank#1{\rule{#1}{0.5pt}}


\usepackage{lipsum}
\usepackage{tikz}
\usepackage{xcolor}

% ==> language
\usepackage[no-math]{fontspec}
\usepackage{fontawesome}
\usepackage[Khmer, Latin]{ucharclasses}
\usepackage{etoolbox}
\usepackage{enumitem}

% ==> main font
\setmainfont[
	BoldFont = Kdam Thmor,
	ItalicFont = Khmer OS Metal Chrieng,
	BoldItalicFont = Khmer OS Muol,
	SmallCapsFont = TeX Gyre Pagella,
	Script=Khmer,Scale=1
]{Khmer OS Siemreap}

\XeTeXlinebreaklocale "kh"
\XeTeXlinebreakskip = 0pt plus 1pt minus 1pt


\newfontfamily{\khmerfamily}
[
	BoldFont = Kdam Thmor,
	ItalicFont = Khmer OS Metal Chrieng,
	BoldItalicFont = Khmer OS Muol,
	SmallCapsFont = TeX Gyre Pagella,
	Script=Khmer,Scale=1
]{Khmer OS Siemreap}
\newfontfamily{\englishfamily}{TeX Gyre Pagella}[Ligatures=TeX, Scale=1.1]
\newfontfamily{\monofamily}[Scale=1.3]{Latin Modern Mono}
\newfontfamily{\latinfamily}[Scale=1.3]{Latin Modern Roman}
\newfontfamily{\tacteingfamily}[Scale=3]{Tacteing}

% Define new font family (have to)   <-- BUG
\newrobustcmd{\englishfont}{\englishfamily\let\currentenglish\englishfamily }
\newrobustcmd{\khmerfont}{\khmerfamily\let\currentkhmer\khmerfamily}
\newrobustcmd{\mono}{\monofamily\let\currentenglish\monofamily }
\newrobustcmd{\en}{\latinfamily\let\currentenglish\latinfamily}
\newrobustcmd{\tacteing}{\tacteingfamily\let\currentenglish\tacteingfamily}

% Initialize fonts
\khmerfont\englishfont
\setTransitionsForLatin{\currentenglish}{\currentkhmer}

% Change typewriter font family
\renewcommand{\ttfamily}{\mono}

% Fixed: math font substitution, if you use |mathpazo|
\let\temp\rmdefault
\usepackage{mathpazo}
\let\rmdefault\temp
% <==
% ==> khmer fonts
\newcommand{\kml}{
	\fontspec[ Script=Khmer, Scale=1,AutoFakeBold=1, AutoFakeSlant=0.25] 
	{Khmer OS Muol}\selectfont
}
\newcommand{\kpali}{
	\fontspec[ Scale=1, Script=Khmer,AutoFakeBold=1, AutoFakeSlant=0.25] 
	{Khmer OS Muol Pali}\selectfont
}
\newcommand{\kbk}{
	\fontspec[Script=Khmer, Scale=1.125,AutoFakeBold=1, AutoFakeSlant=0.25]
	{Khmer OS Bokor}\selectfont
}
\newcommand{\kmetal}{
	\fontspec[Script=Khmer, Scale=1.125,AutoFakeBold=1, AutoFakeSlant=0.25]
	{Khmer OS Metal Chrieng}\selectfont
}


% useful khmer-math shortcuts
\def\KHstop{\quad\text{។}}
\def\KHor{\quad\text{ឬ}\quad}
\def\KHand{\quad\text{និង}\quad}
% <==
% ==> khmer counter
%khmer number

\makeatletter
\def\khmer#1{\expandafter\@khmer\csname c@#1\endcsname}
\def\@khmer#1{\expandafter\@@khmer\number#1\@nil}
\def\@@khmer#1{%
	\ifx#1\@nil% terminate when encounter @nil
	\else%
	\ifcase#1 ០\or ១\or ២\or ៣\or ៤\or ៥\or ៦\or ៧\or ៨\or ៩\fi%
	\expandafter\@@khmer% recursively map the following characters
	\fi}

% khmer alphabet
\def\khmernumeral#1{\@@khmer#1\@nil}
\def\alpkh#1{\expandafter\@alpkh\csname c@#1\endcsname}
\def\@alpkh#1{%
	\ifcase#1% zero -> none
	\or ក\or ខ\or គ\or ឃ\or ង%
	\or ច\or ឆ\or ជ\or ឈ\or ញ%
	\or ដ\or ឋ\or ឌ\or ឍ\or ណ%
	\or ត\or ថ\or ទ\or ធ\or ន%
	\or ប\or ផ\or ព\or ភ\or ម%
	\or យ\or រ\or ល\or វ\or ស%
	\or ហ\or ឡ\or អ%
	\else%[most]
	\@ctrerr % otherwise, counter error!
	\fi}
\makeatother

%declare new enumerate counter
\AddEnumerateCounter{\alpkh}{\@alpkh}{ឈ}
\AddEnumerateCounter{\khmer}{\@khmer}{៣}
% <==

% <==
% ==> redefine the names
\def\chaptername{ជំពូកទី}
\def\axiomName{ស្វ័យសត្យ}
\def\definitionName{និយមន័យ}
\def\theoremName{ទ្រឹស្តីបទ}
\def\lemmaName{បទគន្លឹះ}
\def\propositionName{សំណើ}
\def\corollaryName{វិបាក}
\def\exampleName{ឧទាហរណ៍}
\def\exerciseName{លំហាត់}
\def\proofName{សម្រាយបញ្ជាក់}
\def\solutionName{ដំណោះស្រាយ}
\def\remarkName{សម្គាល់}
% <==
% ==> math
\usepackage{amsmath, amsthm, amsfonts, amssymb}
\usepackage{mathtools}

\mathtoolsset{centercolon} % not work when using |mathpazo|
\DeclarePairedDelimiter\abs{\lvert}{\rvert}
\DeclarePairedDelimiterX\norm[1]\lVert\rVert{
	\ifblank{#1}{\:\cdot\:}{#1}
}
\def\set#1#2{\left\{#1 ~:~ #2\right\}}
\def\permil{\text{\hskip 0.3pt\englishfont\textperthousand}}


\DeclareMathOperator{\arccot}{cot}
\DeclareMathOperator{\arcsec}{arcsec}
\DeclareMathOperator{\arccsc}{arccsc}
\DeclareMathOperator{\lcm}{lcm}
\DeclareMathOperator{\ord}{ord}
\DeclareMathOperator{\sym}{sym}
\DeclareMathOperator{\tr}{trace}
\DeclareMathOperator{\dom}{dom}
\DeclareMathOperator{\ran}{ran}



\def\N{\mathbb{N}}
\def\Z{\mathbb{Z}}
\def\Q{\mathbb{Q}}
\def\Qc{\mathbb{Q}^{\complement}}
\def\R{\mathbb{R}}
\def\C{\mathbb{C}}
\def\F{\mathbb{F}}
\def\K{\mathbb{K}}
\def\P{\mathbb{P}}
\def\labelitemi{$\circ$}
\def\inv{^{-1}}
\def\ang#1{\left\langle#1\right\rangle}
\def\tran{^\mathrm{T}}
\renewcommand{\vec}[1]{\mathbf{#1}}
% <==
% ==> listings
\usepackage{xcolor}
\usepackage{listings}
% ==> basic
\lstset{%
	basicstyle=\small\ttfamily,
	keywordstyle=\color{black},
	commentstyle=\color{gray},
	keywordstyle=[1]{\color{blue!90!black}},
	keywordstyle=[2]{\color{magenta!90!black}},
	keywordstyle=[3]{\color{red!60!orange}},
	keywordstyle=[4]{\color{teal}},
  keywordstyle=[5]{\color{magenta!90!black}}, % <-- otherkeywords, BUG
	commentstyle=\color{gray},
	stringstyle=\color{green!60!black},
	tabsize=2,
	%
	numbers=left,
	numberstyle=\tiny\color{blue!70!gray},
	stepnumber=1,
	%
	frame=Lt,
	breaklines=true,
	xleftmargin=0cm,
	rulecolor=\color{gray!50!black},
	aboveskip=0.5cm,
	belowskip=0.5cm
}
% <==
% ==> code c
\lstdefinelanguage{cmeng}{
  morekeywords={
    auto,break,case,char,const,continue,default,do,double,%
    else,enum,extern,float,for,goto,if,int,long,register,return,%
    short,signed,sizeof,static,struct,switch,typedef,union,unsigned,%
    void,volatile,while},%
  morekeywords=[2]{
    printf, scanf,  include
  },
  otherkeywords={\#,<,>,\&},
  morekeywords=[5]{\#,<,>,\&},
  sensitive,%
	morecomment=[l]{//},
	morecomment=[s]{/*}{*/},
	morestring=[b]',
	morestring=[b]",
}
% <==
% ==> code python
\lstdefinelanguage{py}{
	morekeywords={
		access,and,as,break,class,continue,def,del,elif,else,
		except,exec,finally,for,from,global,if,import,in,is,lambda,
		not,or,pass,print,raise,return,try,while},
	% Built-ins
	morekeywords=[2]{
		abs,all,any,basestring,bin,bool,bytearray,
		callable,chr,classmethod,cmp,compile,complex,delattr,dict,dir,
		divmod,enumerate,eval,execfile,file,filter,float,format,
		frozenset,getattr,globals,hasattr,hash,help,hex,id,input,int,
		isinstance,issubclass,iter,len,list,locals,long,map,max,
		memoryview,min,next,object,oct,open,ord,pow,property,range,
		raw_input,reduce,reload,repr,reversed,round,set,setattr,slice,
		sorted,staticmethod,str,sum,super,tuple,type,unichr,unicode,
		vars,xrange,zip,apply,buffer,coerce,intern,True,False},
	%
	morecomment=[l]\#,%
	morestring=[b]',%
	morestring=[b]",%
	morecomment=[s]{'''}{'''},% used for documentation text
	%                         % (mulitiline strings)
	morecomment=[s]{"""}{"""},% added by Philipp Matthias Hahn
	morestring=[s]{r'}{'},% `raw' strings
	morestring=[s]{r"}{"},%
	morestring=[s]{r'''}{'''},%
	morestring=[s]{r"""}{"""},%
	morestring=[s]{u'}{'},% unicode strings
	morestring=[s]{u"}{"},%
	morestring=[s]{u'''}{'''},%
	morestring=[s]{u"""}{"""},%
	%
	sensitive=true,%
}
% <==
% ==> code asy
\lstdefinelanguage{asy}{ %% Added by Sivmeng HUN
	morekeywords=[1]{
		import, for, if, else,new, do,and, access,
		from, while, break, continue, unravel, 
		operator, include, return},
	morekeywords=[2]{
		struct,typedef,static,public,readable,private,explicit,
		void,bool,int,real,string,var,picture,
		pair, path, pair3, path3, triple, transform, guide, pen, frame
	},
	morekeywords=[3]{
		true,false,and,cycle,controls,tension,atleast,
		curl,null,nullframe,nullpath,
		currentpicture,currentpen,currentprojection,
		inch,inches,cm,mm,pt,bp,up,down,right,left,
		E,N,S,W,NE,NW,SE,SW,
		solid,dashed,dashdotted,longdashed,longdashdotted,
		squarecap,roundcap,extendcap,miterjoin,roundjoin,
		beveljoin,zerowinding,evenodd,invisible
	},
	morekeywords=[4]{
		size,unitsize,draw,dot,label,
		sqrt,sin,cos,tan,cot,Sin,Cos,Tan,Cot,
		graph,
	},
	%
	morecomment=[l]{//},
	morecomment=[s]{/*}{*/},
	morestring=[b]',
	morestring=[b]",
	%
}
% <==
% <==
% ==> mdframe theorem
\usepackage{thmtools}
\usepackage[framemethod=TikZ]{mdframed}

% ==> Axiom
\mdfdefinestyle{mdAxiomStyle}{
	roundcorner = 10pt,
	linewidth=1.125pt,
	skipabove=12pt,
	innerbottommargin=9pt,
	skipbelow=2pt,
	nobreak=true,
	linecolor=blue,
	backgroundcolor=TealBlue!5,
}
\declaretheoremstyle[
	headfont=\bfseries\color{MidnightBlue},
	bodyfont=\itshape,
	mdframed={style=mdAxiomStyle},
	headpunct={\\[3pt]},
	postheadspace={0pt},
	notefont=\bfseries\itshape\color{gray!50!blue},
	notebraces={~(}{)},
]{mdAxiom}
% <==
% ==> Definition
\mdfdefinestyle{mdDefinitionStyle}{
	roundcorner = 10pt,
	linewidth=1.125pt,
	skipabove=12pt,
	innerbottommargin=9pt,
	skipbelow=2pt,
	nobreak=true,
	linecolor=blue,
	backgroundcolor=TealBlue!5,
}
\declaretheoremstyle[
	headfont=\bfseries\color{MidnightBlue},
	bodyfont=\itshape,
	mdframed={style=mdDefinitionStyle},
	headpunct={\\[3pt]},
	postheadspace={0pt},
	notefont=\bfseries\itshape\color{gray!50!blue},
	notebraces={~(}{)},
]{mdDefinition}
% <==
% ==> Theorem
\mdfdefinestyle{mdTheoremStyle}{%
	linewidth=1pt,
	skipabove=12pt,
	frametitleaboveskip=5pt,
	frametitlebelowskip=0pt,
	skipbelow=2pt,
	innertopmargin=5pt,
	innerbottommargin=8pt,
	nobreak=true,
	linecolor=RawSienna!40,
	backgroundcolor=Salmon!10,
}
\declaretheoremstyle[
	headfont=\bfseries\color{RawSienna},
  headindent=0pt,
	bodyfont=\itshape,
	mdframed={style=mdTheoremStyle},
	notefont=\normalfont\itshape\small\color{RawSienna},
	notebraces={~$\big\langle$}{$\big\rangle$},
	headpunct={\\[3pt]},
	postheadspace={0pt},
]{mdTheorem}
% <==
% ==> Exercise
\mdfdefinestyle{mdExerciseStyle}{%
	skipabove=8pt,
	skipbelow=2pt,
	linewidth=1pt,
	rightline=true,
	leftline=true,
	topline=true,
	bottomline=true,
	linecolor=ForestGreen!20,
	backgroundcolor=ForestGreen!20
}
\declaretheoremstyle[
	headfont=\bfseries\color{ForestGreen!70!black},
	bodyfont=\normalfont,
	headindent=0pt,
	spaceabove=2pt,
	spacebelow=2pt,
	mdframed={style=mdExerciseStyle},
	notefont=\small\bfseries\itshape\color{DarkGreen},
	notebraces={~(}{)},
	headpunct={ },
]{mdExercise}
% <==
% ==> Proof
\declaretheoremstyle[
	spaceabove=6pt,
	spacebelow=6pt,
	headindent=0pt,
	headfont=\normalfont\color{magenta!70!blue},
	bodyfont = \normalfont,
	postheadspace=1em,
	qed=$\color{magenta!70!black}\blacksquare$,
	headpunct={ },
]{mdProof}
% <==

% ==> Declaring the theorems
\declaretheorem{axiom}[style=mdAxiom, name={\kml\axiomName}, numberwithin=chapter]
\declaretheorem[style=mdDefinition, name={\kml\definitionName}, sibling=axiom]{definition}

\declaretheorem{theorem}[style=mdTheorem, name={\theoremName},numberwithin=chapter]
\declaretheorem{lemma}[style=mdTheorem, name={\lemmaName},sibling=theorem]
\declaretheorem{proposition}[style=mdTheorem, name={\propositionName}, sibling=theorem]
\declaretheorem{corollary}[style=mdTheorem, sibling=theorem, name={\corollaryName}]

\declaretheorem{example}[name={\exampleName},style=mdExercise, numberwithin=chapter]
\declaretheorem{exercise}[name={\exerciseName},style=mdExercise, sibling=example]

%% Proof
\declaretheorem[name={\bfseries\proofName},numbered=no,style=mdProof]{Proof}
\renewenvironment{proof}{\begin{Proof}}{\end{Proof}}
\declaretheorem[name={\itshape\solutionName},numbered=no,style=mdProof]{solution}
% <==
% <==
% ==> tweaking memmoir
\checkandfixthelayout
%\setlrmarginsandblock{2cm}{8cm}{*}
%\setulmarginsandblock{3cm}{*}{1.5}


\makepagestyle{myruled}
\makeheadrule {myruled}{\textwidth}{\normalrulethickness}
\makefootrule {myruled}{\textwidth}{\normalrulethickness}{\footruleskip}
\makeevenhead {myruled}{}{\small\itshape\leftmark} {}
\makeoddhead  {myruled}{}{\small\itshape\rightmark}{}
\makeevenfoot {myruled}{}{\small\thepage} {}
\makeoddfoot  {myruled}{}{\small\thepage} {}

\makeatletter % because of \@chapapp
\makepsmarks{myruled}{
  \nouppercaseheads
  \createmark{chapter}{both}{shownumber}{}{. \space}
  \createmark{section}{right}{shownumber}{}{. \space}
  \createmark{subsection}{right}{shownumber}{}{. \space}
  \createmark{subsubsection}{right}{shownumber}{}{. \ }
  \createplainmark {toc} {both} {\contentsname} 
  \createplainmark {lof} {both} {\listfigurename}
  \createplainmark {lot} {both} {\listtablename} 
  \createplainmark {bib} {both} {\bibname} 
  \createplainmark {index} {both} {\indexname} 
  \createplainmark {glossary} {both} {\glossaryname}
}
\makeatother
\setsecnumdepth{subsubsection}
\pagestyle{myruled}

\renewcommand{\headheight}{17pt}
% <==



\title{ពិជគណិតលីនេអ៊ែរ}
\author{បង្រៀនដោយ លោកគ្រូ ហាំ ការីម}
\date{\today}

\begin{document}
\maketitle


\chapter{លំហវ៉ិចទ័រ}
\section{រំឭកមេរៀន}
\begin{example}
ប្រមាណវិធី $+$ និង $\times$ ជាប្រមាណវិធីក្នុងលើសំណុំ $\N,~\Z,~\Q,~\R$ ឬ $\C$ ។
ដើម្បីផ្ទៀងផ្ទាត់សំណើខាងលើយើងយក $\K$ ជាសំណុំណាមួយក្នុងចំណោមសំណុំខាងលើ ។
យើងឃើញថាគ្រប់ $a,b\in\K$ យើងបាន
\[
a+b\in\K
\KHand
a\times b\in\K\KHstop
\]
បន្ទាប់មកទៀត យើងនឹងស្រាយលក្ខណៈ uniqueness របស់ប្រមាណវិធីទាំងពីរ ។
យក $a_0,b_0\in\K$ ដែល $a_0=a$ និង $b_0=b$ នោះយើងបាន
\[
a+b=a_0+b_0\KHand
a\times b=a_0\times b_0
\]
ហេតុនេះ $a+b$ និង $ab$ ជាធាតុតែមួយគត់ (unique element) ក្នុងសំណុំ $\K$ ។ 
ដូចនេះប្រមាណវិធី $+$ និង $\times$ ជាប្រមាណវិធីក្នុងលើសំណុំ $\N,~\Z,~\Q,~\R$ ឬ $\C$ ។
\end{example}
\begin{example}
$+$ និង $\times$ មិនមែនជាប្រមាណវិធីក្នុងសំណុំ $\Qc$ ទេព្រោះប្រមាណវិធីនេះមិនស្តាប (closed)
លើ $\Qc$ ឡើយ ។ ជាឧទាហរណ៌យើងអាចយក $a=\sqrt{2}\in\Qc$ និង $b=-\sqrt{2}\in\Qc$ តែ
\[a+b=\sqrt{2}+(-\sqrt{2})=0\notin\Qc\]
ហើយ
\[a\times b=\sqrt{2}(-\sqrt{2})=-2\notin\Qc\]
ដូច្នេះ $+$ និង $\times$ មិនមែនជាប្រមាណវិធីក្នុងលើសំណុំ $\Qc$ ឡើយ ។
\end{example}
\newpage
\begin{example}
ផលគុណស្កាលែនៃពីរវ៉ិចទ័រក្នុង $\R^n$ មិនមែនជាប្រមាណវិធីក្នុងលើ $\R^n$ ទេព្រោះបើយើងយក
$\vec{x}=(x_1,x_2,\dots,x_n)\in\R^n$ និង $\vec{y}=(y_1,y_2,\dots,y_n)\in\R^n$
ដែល $x_i,y_i\in\R$ យើងតែងតែទទួលបាន
\[
\vec{x}\cdot\vec{y}=\sum_{i=1}^{n}x_iy_i\in\R
\]
មានន័យថា $\vec{x}\cdot\vec{y}\notin\R^n$ នោះប្រមាណវិធីនេះមិនមែនជាប្រមាណវិធីក្នុងរបស់សំណុំ
$\R^n$ ទេ ។
\end{example}
\begin{example}
តាង $\mathcal{F}(A,A)$ ជាសំណុំនៃអនុគមន៍ពី $A$ ទៅ $A$ ។ គេកំណត់អនុគមន៍មួយដែល
\[
\begin{array}{lrll}
\circ: & \mathcal{F}(A,A)\times\mathcal{F}(A,A) & \longrightarrow & \mathcal{F}(A,A)\\[0.25cm]
       & (f,g)                                  &\longmapsto      & f\circ g
\end{array}
\]
ជាប្រមាណវិធីក្នុងលើ $\mathcal{F}(A,A)$ ។

ដើម្បីងាយស្រួលសរសេរយើងតាង $\mathcal{F}:=\mathcal{F}(A,A)$ ។
ដើម្បីបង្ហាញថា $\circ$ ជាប្រមាណវិធីក្នុងយើងត្រូវស្រាយឱ្យឃើញពីលក្ខណៈមានតែមួយគត់ (uniqueness)
ហើយមានលក្ខណៈស្តាបលើ $\mathcal{F}$ ។ 
\begin{itemize}
\item លក្ខណៈ Unique ៖ យក $f,g,h,k\in\mathcal{F}$ ដែល $f=h$ និង $g=k$ នោះគ្រប់ $x\in A$ យើងបាន
\begin{align*}
(f\circ g)(x)&=f\Big[g(x)\Big]\\
 &=f\Big[k(x)\Big]   && (\text{ព្រោះ }g=k)\\
 &=h\Big[k(x)\Big]   && (\text{ព្រោះ }f=h)\\
 &=(h\circ k)(x)
\end{align*}
នោះយើងបាន $f\circ g=h\circ k$ ។

\item បន្ទាប់មកយើងស្រាយភាពស្តាបរបស់ $\circ$ ។ ដោយគ្រប់ $f,g\in\mathcal{F}$ នោះ
\[
f\colon A\to A\KHand
g\colon A\to A
\]
គ្រប់ $x\in A$ យើងបាន $f\circ g(x)=f(g(x))\in A$ នោះ
$f\circ g$ ជាអនុគមន៍ពី $A$ ទៅ $A$ ។ មានន័យថា $f\circ g\in\mathcal{F}$ ។
\end{itemize}
ដូចនេះ $\circ$ ជាប្រមាណវិធីក្នុងរបស់សំណុំ $\mathcal{F}$ ។
\end{example}


%% 27-12-2021
\newpage
\begin{definition}
អនុគមន៍ $f:A\to B$ និង $g:A\to B$ ស្មើគ្នាកាលណាដែនកំណត់ $\mathcal{D}(f)=\mathcal{D}(g)$
ហើយ \[f(x)=g(x)\] ចំពោះគ្រប់ $x\in A$ ។
\end{definition}
\begin{example}
បើយើងឱ្យ $f(x)=1+\frac{x^2-1}{x(x-1)}$ និង $g(x)=1+\frac{x+1}{x}$ ។
យើងឃើញថា $f,g$ មើលទៅដូចជា\textit{ស្មើគ្នា}មែន ក៏ប៉ុន្តែដែនកំណត់
\[
\mathcal{D}(f)=\R\setminus\{0,1\}
\KHand
\mathcal{D}(g)=\R\setminus\{0\}
\]
ដូចនេះយើងបាន $f\neq g$ ។
\end{example}

\section{ក្រុម}
\begin{definition}
យក $G$ ជាសំណុំមួយហើយ $*$ ជាប្រមាណវិធីក្នុងរបស់ $G$ ។ គេថា $(G,*)$ ជាក្រុមកាលណា
\begin{itemize}
\item (លក្ខណៈផ្តុំ) $a*(b*c)=(a*b)*c$ ចំពោះគ្រប់ $a,b,c\in G$
\item (ធាតុណឺត) មាន $e\in G$ ដែល $e*a=a*e=a$ ចំពោះគ្រប់ $a\in G$
\item (ធាតុច្រាស) គ្រប់ $a\in G$ មាន $b\in G$ ដែល $a*b=b*a=e$ ។
\end{itemize}
\end{definition}

\newpage
\begin{exercise}
យក $E$ ជាសំណុំមួយ ។ ឧបមាថា $\mathcal{S}(E)$ ជាសំណុំនៃអនុវត្តន៍មួយទល់មួយ (bijection)
ពី $E$ ទៅ $E$ ។ យក $\circ$ ជាប្រមាណវិធីក្នុងលើ $\mathcal{S}(E)$ 
ដែលជាបណ្តាក់រវាងអនុវត្ត៍មួយទល់មួយលើ $E$ ។ តើ $(\S(E), \circ)$ ជាក្រុមឬទេ ? 
ជាក្រុមអែប៊ែលឬទេ ?
\end{exercise}
\begin{solution}
យើងនឹងស្រាយថា ​$(\S(E), \circ)$ ជាក្រុម តែមិនមែនជាក្រុមអាប៊ែលទេ ។ 
ដំបូងយើងត្រូវដឹងថាតើប្រមាណវិធីនេះស្តាបលើ ​$\S(E)$ ឬទេ ។ យក $f,g\in\S(E)$ ។ 
យើងចង់ស្រាយថា $f\circ g$ ក៏ជាអនុវត្តមួយទល់មួយដែរ ។ ឧបមាថាមាន
$x,y\in E$ ដែល $(f\circ g)(x)=(f\circ g)(y)$ នោះយើងបាន
\begin{align*}
&f(g(x))=f(g(y))\\
\implies\quad &g(x)=g(y)  &&\text{(ព្រោះ $f$ មួយទល់មួយ)}\\
\implies\quad &x=y        &&\text{(ព្រោះ $g$ មួយទល់មួយ)}
\end{align*}
ហេតុនេះ $f\circ g\in\S(E)$ ។ ដើម្បីស្រាយថា $\S(E)$ ជាក្រុម យើងត្រូវផ្ទៀងផ្ទាត់លក្ខណៈ ៖
\begin{itemize}
\item យក $f,g,h\in\S(E)$ និង ​$x\in E$ ។ នោះ
\[
((f\circ g)\circ h)x=(f\circ g)(h(x))=f(g(h(x)))
\]
ស្រដៀងគ្នាដែរ យើងបាន
\[
(f\circ (g\circ h))x=f( (g\circ h)x )=f(g(h(x)))
\]
នោះ $(f\circ g)\circ h=f\circ (g\circ h)$ ។
\item តាង $\id\colon E\to E$ ដោយ $id(x)=x$ គ្រប់ $x\in E$ ។ បើ $\id(a)=\id(b)$
នោះយើងបាន $a=b$ ហេតុនេះ $id$ ជាអនុវត្តមួយទល់មួយលើ $E$ ។ យក​ $f\in\S(E)$ 
នោះគ្រប់ $x\in E$ យើងបាន
\[
(f\circ\id)(x)=f(\id(x))=f(x)
\]
ហើយ
\[
(\id\circ f)(x)=\id(f(x))=f(x)
\]
ហេតុនេះ $f\circ\id = \id\circ f=f$ ។ ($\id$ ជាធាតុណឹតក្នុង $\S(E)$) ។

\item យក $f\in\S(E)$ ។ តាមលក្ខណៈរបស់អនុវត្តន៍មួយទល់មួយ យើងបានអនុគមន៍ច្រាស
$f\inv$ ក៏ជាអនុវត្តមួយទល់មួយដែរ ។ ម្យ៉ាងទៀតគ្រប់ $x\in E$ យើងបាន
\[
(f\circ f\inv) (x)=f(f\inv (x))=x=\id(x)
\]
ស្រដៀងគ្នាដែរ
\[
(f\inv\circ f) (x)=f\inv(f(x))=x=\id(x)
\]
នោះ $f\circ f\inv=f\inv\circ f=\id$ (មានន័យថា $f\inv$ ជាធាតុច្រាសនៃ ​$f$) ។
\end{itemize}
ដូចនេះយើងបាន $(\S(E), \circ)$ ជាក្រុម ។ យើងនឹងស្រាយថា $G$ មិនមែនជាក្រុមអាប៊ែលទេ ។
យក $a,b,c\in E$ ជាចំនួនផ្សេងៗគ្នាក្នុង $E$ ។ យើងរើសយកអនុគមន៍ $f,g\in\S(E)$ ដែល
\[
f(a)=b,\quad g(a)=a\KHand g(b)=c
\]
នោះយើងបាន $f(g(a))=f(a)=b$ ក៏ប៉ុន្តែ $g(f(a))=g(b)=c$ ។
ហេតុនេះ $f\circ g\neq g\circ f$ ។ មានន័យថា $(\S(E),\circ)$ មិនមែនជាក្រុមអាប៊ែលទេ ។\\[0.2cm]
ដូចនេះ \fbox{$(\S(E), \circ)$ is a non-Abelian group.}

%តែក្រុមនេះមិនមែនជាក្រុមអាប៊ែលទេ 
%ព្រោះវាជាទូទៅមិនមានលក្ខណៈត្រឡប់ទេ ជាឧទាហរណ៍យើងអាចយក $f,g\in\S(\R)$ កំណត់ដោយ
%$f(x)=x-1$ និង $g(x)=2x$ នោះយើងបាន
%\[
%(f\circ g)(x)=f(g(x))=f(2x)=2x-1
%\]
%ក៏ប៉ុន្តែ
%\[
%(g\circ f)(x)=g(f(x))=g(x-1)=2(x-1)=2x-2\KHstop
%\]
%ដូចនេះ \quad\fbox{$(\S(E), \circ)$ ជា non-Abelian group ។}
\end{solution}

\begin{corollary}
គេយក ​$n\in\N$ ហើយគេយក $E=\{1,2,3,\dots,n\}$ ។ ដូចលំហាត់ខាងលើដែរ យើងកំណត់សរសេរ
$\S_n:=\S(E)$ ជាសំណុំនៃអនុគមន៍មួយទល់មួយពី $E$ ទៅ $E$ ។ យើងបាន $(\S_n,\circ)$​ ជាក្រុម
តែមិនមែនជាក្រុមអាប៊ែលទេ (non-Abelian group) ។
\end{corollary}

\begin{proposition}
យក $(G,*)$ ជាក្រុម ។
\begin{enumerate}[label={(\alph*)}]
	\item ធាតុណឺតមានតែមួយគត់
	\item ធាតុច្រាស $x'$ របស់ $x$ មានតែមួយគត់
	\item ធាតុច្រាស់របស់ចម្រាសរបស់ $x$ គឺ $x$; មានន័យថា $(x')'=x$
	\item គ្រប់ធាតុ $x,y\in G$ យើងបាន $(x*y)=y'*x'$
	\item គ្រប់ធាតុ $x,y,z\in G$ បើ $x*y=x*z$ នោះ $y=z$ ។
\end{enumerate}
\end{proposition}
\begin{proof}
\text{}
\begin{enumerate}[label={(\alph*)}]
	\item យើងឧបមាថាមាន $e,e_0$ ជាធាតុណឺតរបស់ $G$ ។
	ដោយ $e_0$ ជាធាតុណឺតនោះយើងបាន $e_0*e=e*e_0=e$ ។ ដូចគ្នាដែរ ដោយ $e$ ជាធាតុណឺតនោះ
	\[e_0*e=e*e_0=e_0\]
	នោះយើងបាន $e=e_0$ ។ ដូចនេះធាតុណឺតមានតែមួយគត់ ។
	%%
	\item ស្រដៀងគ្នាដែរ យើងឧបមាថាមាន $a,b$ ជាធាតុច្រាសរបស់ $x$ ។ ហេតុនេះ
	\begin{align*}
	a=e*a=(b*x)*a=b*(x*a)=b*e=b
	\end{align*}
	នោះ $a=b$ ។ ដូចនេះធាតុណឹតរបស់ $x$ មានតែមួយគត់ហើយយើងនឹងតាងវាដោយ $x'$ ។
	%%
	\item ដោយហេតុថា $x'$ ជាធាតុច្រាសនៃ $x$ នោះយើងបាន
	\[x*x'=x'*x=e\]
	នោះយើងបានធាតុច្រាសរបស់ $x'$ គឹ $x$ ។ ដូចនេះ $\boxed{(x')'=x}$ ។
	%%
	\item យក $x,y\in G$ នោះ $x',y'\in G$ (well defined) ។ ពិនិត្យមើល
	\[(x*y)*(y'*x')=x*(y*y')*x'=x*e*x'=e\]
	ម្យ៉ាងទៀត
	\[(y'*x')*(x*y)=y'*(x'*x)*y=y'*e*y=e\]
	នោះយើងទាញបានថាចម្រាសរបស់ $x*y$ គឹ $y'*x'$ ។ \\[0.2cm]
	ដូចនេះ $\boxed{(x*y)'=y'*x'}$ ។
	%%
	\item យើងមាន $x,y,z\in G$ ដែល $x*y=x*z$ ។ ដោយ $*$ ជាប្រមាណវិធីក្នុងរបស់ $G$ នោះ
	\begin{align*}
	&x'*(x*y)=x'*(x*z)\\ \implies\quad
	&(x'*x)*y=(x'*x)*z\\ \implies\quad
	&e*y=e*z \\ \implies\quad
	&y=z
	\end{align*}
	ម្យ៉ាងទៀត ចម្រាសនៃសំណើនេះ (converse of this statement) ក៏ពិតដែរ \\[0.2cm]
	ដូចនេះ \fbox{$x*y=x*z\iff y=z$} ។
\end{enumerate}
\end{proof}












\end{document}
